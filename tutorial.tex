%% abtex2-modelo-artigo.tex, v-1.9.7 laurocesar
%% Copyright 2012-2018 by abnTeX2 group at http://www.abntex.net.br/ 
%%
%% This work may be distributed and/or modified under the
%% conditions of the LaTeX Project Public License, either version 1.3
%% of this license or (at your option) any later version.
%% The latest version of this license is in
%%   http://www.latex-project.org/lppl.txt
%% and version 1.3 or later is part of all distributions of LaTeX
%% version 2005/12/01 or later.
%%
%% This work has the LPPL maintenance status `maintained'.
%% 
%% The Current Maintainer of this work is the abnTeX2 team, led
%% by Lauro César Araujo. Further information are available on 
%% http://www.abntex.net.br/
%%
%% This work consists of the files abntex2-modelo-artigo.tex and
%% abntex2-modelo-references.bib
%%

% ------------------------------------------------------------------------
% ------------------------------------------------------------------------
% abnTeX2: Modelo de Artigo Acadêmico em conformidade com
% ABNT NBR 6022:2018: Informação e documentação - Artigo em publicação 
% periódica científica - Apresentação
% ------------------------------------------------------------------------
% ------------------------------------------------------------------------

\documentclass[
% -- opções da classe memoir --
article,			% indica que é um artigo acadêmico
11pt,				% tamanho da fonte
oneside,			% para impressão apenas no recto. Oposto a twoside
a4paper,			% tamanho do papel. 
% -- opções da classe abntex2 --
%chapter=TITLE,		% títulos de capítulos convertidos em letras maiúsculas
%section=TITLE,		% títulos de seções convertidos em letras maiúsculas
%subsection=TITLE,	% títulos de subseções convertidos em letras maiúsculas
%subsubsection=TITLE % títulos de subsubseções convertidos em letras maiúsculas
% -- opções do pacote babel --
english,			% idioma adicional para hifenização
brazil,				% o último idioma é o principal do documento
sumario=tradicional
]{abntex2}


% ---
% PACOTES
% ---

% ---
% Pacotes fundamentais 
% ---
\usepackage{lmodern}			% Usa a fonte Latin Modern
\usepackage[T1]{fontenc}		% Selecao de codigos de fonte.
\usepackage[utf8]{inputenc}		% Codificacao do documento (conversão automática dos acentos)
\usepackage{indentfirst}		% Indenta o primeiro parágrafo de cada seção.
\usepackage{nomencl} 			% Lista de simbolos
\usepackage{color}				% Controle das cores
\usepackage{graphicx}			% Inclusão de gráficos
\usepackage{microtype} 			% para melhorias de justificação
% ---

% ---
% Pacotes adicionais, usados apenas no âmbito do Modelo Canônico do abnteX2
% ---
\usepackage{lipsum}				% para geração de dummy text
% ---

% ---
% Pacotes de citações
% ---
\usepackage[brazilian,hyperpageref]{backref}	 % Paginas com as citações na bibl
\usepackage[alf]{abntex2cite}	% Citações padrão ABNT
% ---

% ---
% Configurações do pacote backref
% Usado sem a opção hyperpageref de backref
\renewcommand{\backrefpagesname}{Citado na(s) página(s):~}
% Texto padrão antes do número das páginas
\renewcommand{\backref}{}
% Define os textos da citação
\renewcommand*{\backrefalt}[4]{
	\ifcase #1 %
	Nenhuma citação no texto.%
	\or
	Citado na página #2.%
	\else
	Citado #1 vezes nas páginas #2.%
	\fi}%
% ---

% --- Informações de dados para CAPA e FOLHA DE ROSTO ---
\titulo{Tutorial de Desenvolvimento com FrameWeb SPA}
\tituloestrangeiro{FrameWeb SPA Development Tutorial}

\autor{
	Pedro Henrique Brunoro Hoppe\thanks{``Mestrando.'' \url{pedrohoppe@gmail.com}} 
	\\[0.5cm] 
	Vitór Estêvao Silva Souza\thanks{``Constar currículo sucinto de cada autor, com 
		vinculação corporativa e endereço de contato.''} }

\local{Brasil}
\data{2022}
% ---

% ---
% Configurações de aparência do PDF final

% alterando o aspecto da cor azul
\definecolor{blue}{RGB}{41,5,195}

% informações do PDF
\makeatletter
\hypersetup{
	%pagebackref=true,
	pdftitle={\@title}, 
	pdfauthor={\@author},
	pdfsubject={Modelo de artigo científico com abnTeX2},
	pdfcreator={LaTeX with abnTeX2},
	pdfkeywords={abnt}{latex}{abntex}{abntex2}{atigo científico}, 
	colorlinks=true,       		% false: boxed links; true: colored links
	linkcolor=blue,          	% color of internal links
	citecolor=blue,        		% color of links to bibliography
	filecolor=magenta,      		% color of file links
	urlcolor=blue,
	bookmarksdepth=4
}
\makeatother
% --- 

% ---
% compila o indice
% ---
\makeindex
% ---

% ---
% Altera as margens padrões
% ---
\setlrmarginsandblock{3cm}{3cm}{*}
\setulmarginsandblock{3cm}{3cm}{*}
\checkandfixthelayout
% ---

% --- 
% Espaçamentos entre linhas e parágrafos 
% --- 

% O tamanho do parágrafo é dado por:
\setlength{\parindent}{1.3cm}

% Controle do espaçamento entre um parágrafo e outro:
\setlength{\parskip}{0.2cm}  % tente também \onelineskip

% Espaçamento simples
\SingleSpacing


% ----
% Início do documento
% ----
\begin{document}
	
	% Seleciona o idioma do documento (conforme pacotes do babel)
	%\selectlanguage{english}
	\selectlanguage{brazil}
	
	% Retira espaço extra obsoleto entre as frases.
	\frenchspacing 
	
	% ----------------------------------------------------------
	% ELEMENTOS PRÉ-TEXTUAIS
	% ----------------------------------------------------------
	
	%---
	%
	% Se desejar escrever o artigo em duas colunas, descomente a linha abaixo
	% e a linha com o texto ``FIM DE ARTIGO EM DUAS COLUNAS''.
	% \twocolumn[    		% INICIO DE ARTIGO EM DUAS COLUNAS
	%
	%---
	
	% página de titulo principal (obrigatório)
	\maketitle
	
	
	% titulo em outro idioma (opcional)
	
	% ----------------------------------------------------------
	% ELEMENTOS TEXTUAIS
	% ----------------------------------------------------------
	\textual
	
	% ----------------------------------------------------------
	% Introdução
	% ----------------------------------------------------------
\section{Instalação}
	
\subsection{Instalação do FrameWeb SPA}
Para instalar o FrameWeb SPA acesse o sítio eletrônico \url{https://github.com/pedrohbh/FrameWebSPA} e clone o repositório para sua máquina (ou baixe o zip disponível em ``Code'' -> ``Download Zip'') [\autoref{fig:tutorial-1}].

\begin{figure}
	\centering
	\includegraphics[width=\linewidth]{"figuras/Tutorial 1"}
	\caption{Repositório GitHub}
	\label{fig:tutorial-1}
\end{figure}

Após baixar o repositório, siga os seguintes passos:
\begin{enumerate}
	\item Abra o Eclipse;
	\item Clique no menu Help -> Install New Software...;
	\item Clique em ``Add...'' (\autoref{fig:siriustuto516});
	\item Preencha os campos:
	\begin{description}
		\item[Name:] Pode ser o nome que você quiser. \textit{Sugestão: FrameWeb SPA};
		\item[Location:] Clique no botão ``Archive…'' e depois selecione o arquivo ``FrameWeb.zip'' na pasta onde foi clonado o repositório do GitHub anteriormente;
	\end{description}
	\item Clique no botão ``Add'' (\autoref{fig:tutorial-2});
	\item Em ``Work with:'' selecione o repositório adicionado no passo anterior (no nosso caso, \textit{FrameWeb SPA}), marque as opções ``FrameWeb Code Generator'' e ``FrameWeb Graphical Editor'' e clique em ``Next >'' (\autoref{fig:tutorial-3});
	\item Siga os demais passos de instalação aceitando os termos de licença (\autoref{fig:licenca}). Caso apareça uma mensagem dizendo que os certificados de segurança não são reconhecidos, apenas marque todos e clique em Ok (\autoref{fig:certificado});
	\item Após a instalação, reinicie o Eclipse.
\end{enumerate}

\begin{figure}
	\centering
	\includegraphics[width=0.7\linewidth]{figuras/Sirius_tuto5_16}
	\caption{Menu de instalação do Eclipse}
	\label{fig:siriustuto516}
\end{figure}

\begin{figure}
	\centering
	\includegraphics[width=0.7\linewidth]{"figuras/Tutorial 2"}
	\caption{Passo 5}
	\label{fig:tutorial-2}
\end{figure}

\begin{figure}
	\centering
	\includegraphics[width=0.7\linewidth]{"figuras/Tutorial 3"}
	\caption{Passo 6}
	\label{fig:tutorial-3}
\end{figure}

\begin{figure}
	\centering
	\includegraphics[width=0.7\linewidth]{figuras/licenca}
	\caption{Licença}
	\label{fig:licenca}
\end{figure}

\begin{figure}
	\centering
	\includegraphics[width=0.7\linewidth]{figuras/certificado}
	\caption{Certificado}
	\label{fig:certificado}
\end{figure}


Após a instalação, a montagem do seu projeto segue os mesmos passos disponíveis na Wiki do FrameWeb \url{https://github.com/nemo-ufes/FrameWeb/wiki/ToolsTutorial02}. A \textbf{exceção} é a inclusão do projeto de arquitetura do seu framework, que ao invés de você baixar o JButler da página Github, você usará um dos três frameworks, a saber, Angular, React ou VueJS que estão disponíveis na pasta ``Frameworks'' no repositório clonado anteriormente \url{https://github.com/pedrohbh/FrameWebSPA}. Importe também a definição da linguagem Java dentro da pasta ``Language''.


%Open Eclipse;
%Click on the menu Help > Install New Software...
%In the Work with: field, type http://dev.nemo.inf.ufes.br/framewebplugin/ and press Enter;
%Unmark the checkbox Group items by category;
%Mark the checkbox for all FrameWeb Tools that show in the list (see figure below);
%Click Next twice, select the I accept the terms of the license agreement option and then click Finish;
%After the installation, restart Eclipse.



\section{Desenvolvimento}
O desenvolvimento do projeto seguem as mesmas instruções da Wiki \url{https://github.com/nemo-ufes/FrameWeb/wiki} para todos os modelos. A exceção é o Modelo de Navegação que também segue as mesmas instruções, porém com algumas mudanças que serão descritas a seguir.
	
\subsection{Partial}

O elemento novo principal é o Partial. Ele representa a parte gráfica de um \textit{component} SPA. Possui as mesmas características do elemento <<page>> para os Frameworks MVC tradicionais, porém sua interpretação (inclusive para o gerador de código) difere de <<Page>> e para o caso dos frameworks SPA, o Partial deve ser utilizado ao invés de Page.

\subsection{Navigation Aggregation Association}	

	\section{Considerações finais}
	
	\lipsum[1]
	
	\begin{citacao}
		\lipsum[2]
	\end{citacao}
	
	\lipsum[3]
	
	% ----------------------------------------------------------
	% ELEMENTOS PÓS-TEXTUAIS
	% ----------------------------------------------------------
	\postextual
	
	% ----------------------------------------------------------
	% Referências bibliográficas
	% ----------------------------------------------------------
	\bibliography{abntex2-modelo-references}
	
	% ----------------------------------------------------------
	% Glossário
	% ----------------------------------------------------------
	%
	% Há diversas soluções prontas para glossário em LaTeX. 
	% Consulte o manual do abnTeX2 para obter sugestões.
	%
	%\glossary
	
	% ----------------------------------------------------------
	% Apêndices
	% ----------------------------------------------------------
	
	% ---
	% Inicia os apêndices
	% ---
	\begin{apendicesenv}
		
		% ----------------------------------------------------------
		\chapter{Nullam elementum urna vel imperdiet sodales elit ipsum pharetra ligula
			ac pretium ante justo a nulla curabitur tristique arcu eu metus}
		% ----------------------------------------------------------
		\lipsum[55-56]
		
	\end{apendicesenv}
	% ---
	
	% ----------------------------------------------------------
	% Anexos
	% ----------------------------------------------------------
	\cftinserthook{toc}{AAA}
	% ---
	% Inicia os anexos
	% ---
	%\anexos
	\begin{anexosenv}
		
		% ---
		\chapter{Cras non urna sed feugiat cum sociis natoque penatibus et magnis dis
			parturient montes nascetur ridiculus mus}
		% ---
		
		\lipsum[31]
		
	\end{anexosenv}
	
	% ----------------------------------------------------------
	% Agradecimentos
	% ----------------------------------------------------------
	
	\section*{Agradecimentos}
	Texto sucinto aprovado pelo periódico em que será publicado. Último 
	elemento pós-textual.
	
\end{document}
